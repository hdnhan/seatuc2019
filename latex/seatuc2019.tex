% 6 pages max
\documentclass[twocolumn]{article}
% packages
%\usepackage[utf8]{vietnam}


% ---------------------------------------------
% perso : packages and macros
% ---------------------------------------------
\usepackage{latexsym}
\usepackage{mathrsfs}
%\usepackage{amsfonts}
\usepackage{amssymb}
\usepackage{hyperref}
\hypersetup{
	colorlinks=true, %colorise les liens
	breaklinks=true, %permet le retour a la ligne dans les liens trop longs
	urlcolor=blue, %couleur des hyperliens
	linkcolor=blue, %couleur des liens internes
	citecolor=blue, %couleur des references
	pdftitle={An implicit high order discontinuous
		Galerkin level set method
		for two-phase flow problems}, %informations apparaissant dans
	pdfauthor={Ta Pigeonneau Saramito}, %les informations du document
	pdfsubject={level set} %sous Acrobat.
}
\newcommand{\Frac}[2] {\displaystyle \frac{#1}{#2}}
\newcommand{\red}[1]  {{\color[rgb]{1,0,0}{{#1}}}}
\renewcommand{\leq}{\leqslant}
\renewcommand{\geq}{\geqslant}
\newcommand{\jump}[1]    {[\![{#1}]\!]}
\newcommand{\average}[1] {\{\!\!\{{#1}\}\!\!\}}




\usepackage{amsmath, amssymb, amsthm}
%\usepackage{color, graphicx, cases}
%\usepackage{hyperref}
%\hypersetup{
%	colorlinks=true,
%	linkcolor=black,
%	filecolor=black,      
%	urlcolor=black,
%}
%\usepackage{array, multirow, booktabs}
%\usepackage{caption, subcaption}
%\usepackage{ragged2e} % using justifying
\numberwithin{equation}{section}
\everymath{\displaystyle}
%\usepackage{titling}

%\usepackage[a4paper,left=35mm,top=31mm,right=20mm,bottom=30mm]{geometry}
%\renewcommand{\baselinestretch}{1.5}

% new definitions
\newtheorem{dl}{Theorem}[section]
\newtheorem{md}{Proposition}[section]
\newtheorem{hq}{Corollary}[section]
\newtheorem{cy}{Remark}[section]

\theoremstyle{definition}
\newtheorem{dn}{Definition}[section]
\newtheorem{vd}{Example}[section]
\newtheorem{bt}{Problem}[section]
\newtheorem{nx}{Comment}[section]

% reference
\usepackage{cleveref}
\crefname{dl}{\textbf{Theorem}}{}
\crefname{md}{\textbf{Proposition}}{}
\crefname{hq}{\textbf{Corollary}}{}

\crefname{dn}{\textbf{Definition}}{}
\crefname{vd}{\textbf{Example}}{}
\crefname{cy}{\textbf{Remark}}{}
\crefname{bt}{\textbf{Problems}}{}
\crefname{nx}{\textbf{Comment}}{} 

\title{\bf Numerical simulation of heat transfer problem by Freefem++ software}
\author{Ta Thi Thanh Mai\thanks{email: mai.tathithanh@hust.edu.vn}\and Ho Duc Nhan\thanks{email: hdnhan28@gmail.com}\and Tran Minh Tam\thanks{email: tam.tranminh22@gmail.com}}
%\thanksmarkseries{arabic}
\date{\footnotesize\textit{School of Applied Mathematics and Informatics, Hanoi University of Science and Technology, No.1 Dai Co Viet street, Hai Ba Trung District, Hanoi, Vietnam}}
\justifying
% ---------------------------------------------
\begin{document}
\twocolumn[
	\maketitle
	\begin{@twocolumnfalse}
		\rule{\textwidth}{.1pt}
		\begin{abstract}
		We study the problem of heat transfer and its application in engineering. The main purpose of this paper is to present a numerical scheme for parabolic equations in FreeFem++ and implement some simulations of heat equation in shape design and optimal control problem. We propose a variational method in combination with continuous Galerkin finite element methods and an implicit scheme for discretization in time. A numerical scheme with error estimate are given for direct problem. The optimal control problem is solved by using IPOPT optimizer. Some numerical examples are investigated for showing the efficiency and accuracy of present solver.
		\end{abstract}
		\vspace{1ex}
		\noindent \textit{Keywords: Crank-Nicolson, parabolic equations, backward Euleur, Galerkin finite element, optimal control.
		}
		\\
		\rule{\textwidth}{.1pt}
		\vspace{2mm}
	\end{@twocolumnfalse}
]
\saythanks

% --------------------------------------------------------------------
\section{Introduction}
\quad Heat transfer via conduction in solid objects is an important part in thermal engineering and solid mechanics. Its mathematical expression, the parabolic equation has received a large amount of attention from engineers and mathematicians. Despite a vast literature on numerical solving of this problem, it still requires further investigation to reach challenging modelings. 

In this article, we aim to construct a numerical scheme for parabolic equation and apply it to simulate one case of thermal engineering and optimal control problems. We focus on numerical simulations by using Freefem++ \url{https://freefem.org}, an efficient tool for solving PDE equations and visualization software medit \cite{Fre01}. The main contribution of this work is giving a solver of heat equation with exhaustive study for error convergence. This is an important feature to simulate more complex modeling in thermal engineering and other problem.

This paper is organized into five sections. The next section 
describes our approach of heat equation by a common boundary value problem. Section 3 presents the numerical method: a variational formulation with an implicit scheme for time discretization and finite element methods for space discretization. All the numerical test cases are outlined in Section 4.  The last section gives some perspectives and comments about the effectiveness and reliability of the present scheme.
\section{Setting of problem}
% --------------------------------------------------------------------
\quad Let $\Omega \subset \mathbb{R}^d,\, d\in \mathbb{N^+}$ be a bounded domain with a boundary $\Gamma$ and denote the cylinder $Q=\Omega\times (0,\, T]$ and the lateral surface area $S=\Gamma \times [0,\, T]$ where $T>0$. 

Consider the heat equation:
\begin{align}\label{1.1}
	\frac{\partial u(x, t)}{\partial t}+\mathcal{L}u(x, t)=F(x, t), \quad(x, t)\in Q,
\end{align}
with the Dirichlet boundary and initial conditions, respectively
\begin{align}
	u(x, t)&=0,\quad(x, t)\in S, \label{1.2}\\
	u(x, 0)&=u_0(x),\quad\quad\, x\in \Omega,\label{1.3}
\end{align}
where
\begin{align*}
	&\mathcal{L}u = -\sum_{i, j=1}^{d}\frac{\partial}{\partial x_i}\left(a_{ji}\frac{\partial u}{\partial x_j}\right)+a_0u,\\
	&a_{ji}\in L^{\infty}(Q),\, a_{ij}=a_{ji},\; \forall i, j\in \{1, 2, ..., d\},\\
	&\lambda_1\left\|\xi\right\|^2\leq \sum_{i, j=1}^{d}a_{ij}\xi_i\xi_j\leq \lambda_2\left\|\xi\right\|^2,\; \forall \xi\in\mathbb{R}^d,\\
	&a_0\in L^{\infty}(Q),\; 0\leq a_0(x, t)\leq \mu_1,\; (x, t)\in Q,\\ 
	&u_0\in L^2(\Omega),%\;u_D\in L^2(S),
\end{align*}
with $\lambda_1$ and $\lambda_2$ are positive constants and $\mu_1\geq 0$. 

The problem is that to determine $u(x, t)$ when all data $a_{ji}(x, t),\,a_0(x, t),\,u_0(x)$ and $F(x, t)$ in equations \eqref{1.1} - \eqref{1.2} - \eqref{1.3} are given called direct problem.
% --------------------------------------------------------------------
\section{Numerical method} 
% --------------------------------------------------------------------
\subsection{Variational problem}
% --------------------------------------------------------------------
\quad Multiplying \eqref{1.1} by an efficient smooth test function $v$, integrating over $\Omega$ and then applying Green's formula, see \cite{F05}, leads to the problem: Find $u(.,t)\in H^1_0(\Omega)$ such that
\begin{align}\label{2.1}
	\left\langle \frac{\partial u}{\partial t}, v \right\rangle+a\left(u, v\right)=\left\langle F, v \right\rangle,\; \forall v\in H^1_0(\Omega),
\end{align} 
\begin{align}\label{2.2}
	u(x, 0)=u_0(x), \; x\in \Omega,
\end{align}
where 
$$a\left(u, v\right)=\int_{\Omega}\left[\sum_{i, j=1}^{d}a_{ji}\frac{\partial u}{\partial x_i}\frac{\partial v}{\partial x_j}+a_0uv\right]dx,$$
$$\left\langle F, v \right\rangle=\int_{\Omega}F vdx.$$

Following \cite{L98, W87, C87, SV03}, we can prove that there exists a unique solution of the problem \eqref{2.1} - \eqref{2.2}. We approximate this solution by finite element method as follows. 
\subsection{Finite element method}
\quad Now we present a fully discrete finite element approximation for the variational problem \eqref{2.1}.
\begin{itemize}
	\item For \textit{spatial approximation}, let $\mathcal{T}_h$ be a triangulation of $\Omega$ and define a piecewise linear finite element space $V_h \subset H^1_0(\Omega)$ by
	$$V_h=\left\{v_h:v_h\in C(\overline{\Omega}), v_h|_K\in P_1(K), \forall K\in \mathcal{T}_h\right\},$$
	where $P_1(K)$ is a continuous piecewise linear polynomial on the element $K$.
	\item For \textit{time discretization}, discrete $[0, T]$ uniformly into $M$ steps, where $t_n=n\Delta t,\, n=0, 1, \dots, M$ with the time step size $\Delta t = T/M$. Denote a function $u(x, t_n)=u^n(x)$.
\end{itemize}

Therefore, the problem is to find $u^n_h\in V_h$ for $n=1, 2, \dots, M$ with $\theta \in [0, 1]$ such that
\begin{align}\label{2.3}
	&\langle d_tu^n_h, v_h \rangle+a\left(\theta u^n_h+(1-\theta)u^{n-1}_h, v_h\right)\notag\\
	&\qquad\quad=\left\langle \theta F^n+(1-\theta)F^{n-1}, v_h \right\rangle, \; \forall v_h\in V_h,
\end{align}
and the initial condition 
\begin{align}\label{2.4}
	u^0_h=u_0,
\end{align}
where $d_tu^n_h=\frac{u^n_h-u^{n-1}_h}{\Delta t}, \; n=1, 2, ..., M$.

We have different methods depending on $\theta$ such as backward Euler ($\theta=1$) and Crank-Nicolson ($\theta=0.5$). The discrete variational problem \eqref{2.3} -\eqref{2.4} admits a unique solution $u^n_h\in V_h$. Let $u_h(x, t)$ be the linear interpolation of $u_h^n$ with respect to $t$. Therefore, for $x\in \Omega,\, t\in [t_{n-1}, t_n]$, we have
\begin{align*}
	u_h(x, t)=\frac{t-t_{n-1}}{\Delta t}u_h^{n-1}+\frac{t_n-t}{\Delta t}u_h^{n}.
\end{align*}
\begin{dl}\label{dl2.1}
	Let $u(x, t)$ be the solution of variational problem \eqref{2.1} - \eqref{2.2} and $u^n_h\in V_h$ for $n=1, 2, \dots, M$ be the solution for \eqref{2.3} - \eqref{2.4}. Then there holds the error estimate, see \cite{C87}
	\begin{align}\label{2.5}
		\left\|u_h-u\right\|_{L^2(Q)}=
		\begin{cases}
			O\left(h^2+\Delta t\right), &\theta=\{1\},\\
			O\left(h^2+\Delta t^2\right), &\theta=\{0.5\},
		\end{cases}		
	\end{align}
	where $h$ is the mesh size.
\end{dl}


% --------------------------------------------------------------------
\section{Tests and discussion}
%In this section, we investigate some numerical test to evaluate the accuracy and ability of the method. The important questions is how 
% software such as Ansys, Solidwork,... provide simulation features of heat transfer, but they lack of mathematical clarity for mathematicians.
% --------------------------------------------------------------------
\subsection{Error evaluation with exact solution}
\quad We study a numerical experiment with the exact solution of heat equation and evaluate the error convergence. Consider a square $\Omega = [0,1]^2$. Find $u(x, t)$ satisfying
\begin{equation}
	\dfrac{\partial u}{\partial t} - \left(\dfrac{\partial^2 u}{\partial x_1^2} + \dfrac{\partial^2u}{\partial x_2^2}\right) = (1+2\pi^2)e^t\sin(\pi x_1) \sin(\pi x_2),
\end{equation}
with the initial and boundary conditions
$$ u(x, 0) = \sin(\pi x_1)\sin(\pi x_2) \quad \text{and} \quad u|_S = 0.$$
The exact solution is
$$u(x, t) = e^t\sin(\pi x_1) \sin(\pi x_2), $$

In order to quantify the difference between the approximate solution and the exact one, different cases of mesh size and time step length were studied to show the convergence of $L^2$-error. Fig. \ref{fig:time} shows, for this error measurement, the convergence of the approximate solution to the exact one when the time step tends to zero. Remark that, for each $h$, the error tends to a constant that is independent of time step. Numerically, we are looking for a representation of the error as the sum of two terms, one depends on $h$ and the other depend on $\Delta t$: $\left\|u_h-u\right\|_{L^2(Q)}=O\left(h^{\alpha}+\Delta t^{p}\right)$. Fig \ref{fig:mesh} shows the error versus $h$ for various mesh size, the time step $\Delta t$ has been chosen sufficiently small such that the error depends only upon $h$. Observe that the approximate solution converges to the exact one with mesh refinement with a power index $\alpha \approx 2$. This result is in good accordance with the Theorem \ref{dl2.1} in the previous section.
\begin{figure}[h!]
	\centering
	\begin{tabular}{c}
		\includegraphics[width=.8\linewidth]{figures/CNt}
	\end{tabular}
	\caption{$L^2$ error convergence of Crank-Nicolson scheme depends on time step.}
	\label{fig:time}
\end{figure}
\begin{figure}[h!]
	\centering
	\begin{tabular}{c}
		\includegraphics[width=.8\linewidth]{figures/CN}
	\end{tabular}
	\caption{$L^2$ error convergence of Crank-Nicolson scheme depends on mesh size.}
	\label{fig:mesh}
\end{figure}

\subsection{A problem of thermal engineering}
\quad The parameters $a_{ji}$ in \eqref{1.1} describe thermal conductivity property of a specific object. For a homogeneity object, thermal conductions are equal for all dimensions, which mean $a_{ji}=const \quad \forall i,j=\overline{1,d}$. The constant is thermal conductivity coefficient of the object material. In two dimensional case, \eqref{1.1} forms as
\begin{align}\label{4.2}
	\dfrac{\partial u}{\partial t} - \kappa \left(\dfrac{\partial^2 u}{\partial x_1^2}+\dfrac{\partial^2 u}{\partial x_2^2}\right) = F.
\end{align}
\quad We apply the numerical simulations of heat transfer into designing heat sink. Assuming a hot CPU inside a rectangular room fill with air. Let $u=u_{hot}$ inside CPU region and $u=u_{air}$ on air region, respectively $\Omega_{c}$ and $\Omega_{a}$, on the initial time. Our goal is to design a heat sink stick on the CPU to lower its temperature. Figure \ref{fig:testsink} describes the setting of this experiment.\\
\begin{figure}[ht]
	\centering
	\begin{tikzpicture}
		\draw (0,0) -- (6,0);
		\draw (6,0) -- (6,4);
		\draw (6,4) -- (0,4);
		\draw (0,4) -- (0,0);
		\draw (4,0) -- (4,0.8);
		\draw (4,0.8) -- (2,0.8);
		\draw (2,0.8) -- (2,0);
		\node at (3,0.4) {$\Omega_c$};
		\node at (3,2.5) {$\Omega_a$};
		\node at (3,1.1) {$\Omega_s$};
		\draw (2,1.5) .. controls (2.7,2) and (3.4,1) .. (4,1.2);
		\draw (2,0.8) -- (2,1.5);
		\draw (4,0.8) -- (4,1.2);
	\end{tikzpicture}
	\caption{Illustration of simple cooling system for CPU.}
	\label{fig:testsink}
\end{figure}
\newpage
The heat sink region, denoted by $\Omega_{s}$, has thermal conductivity coefficient $\kappa_s$. Similarly, let $\kappa_a$ and $\kappa_c$ be respectively the thermal conductivity coefficients inside air and CPU region. Technically, $\kappa_a$ value is small compared to $\kappa_c$ and $\kappa_s$ due to thermal conducting nature of air. Furthermore, to provide cooling ability, $\kappa_s>\kappa_c$. The visualization using medit software. At initial state, set $u_{hot}=80$ and $u_{air}=20$ as in Figure \ref{fig:ini}.\\
\begin{figure}[http]
	\centering
	\begin{tabular}{c}
		\includegraphics[width=.8\linewidth]{figures/begin}
	\end{tabular}
	\caption{Thermal distribution at initial state. $u|_{\Omega_{c}} = 80$, $u|_{\Omega_{a}\cup\Omega_{s}}=20$.}
	\label{fig:ini}
\end{figure}\\
Set $T=1s$, $\kappa_a=0.01$, $\kappa_c=1$, $\kappa_s=100$. The thermal distribution, maximum and minimum temperature inside domain $\Omega_{c}$  at final time $T$ of different heat sink shapes are shown in Figures \ref{fig:nosink}, \ref{fig:recsink} and \ref{fig:finsink}.\\
\begin{figure}[http]
	\centering
	\begin{tabular}{c}
		\includegraphics[width=.8\linewidth]{figures/nosinkc} \\ \includegraphics[width=.8\linewidth]{figures/nosinkb}
	\end{tabular}
	\caption{Thermal conduction without heat sink. $u_{min}=63.1$, $u_{max}=67.3$.}
	\label{fig:nosink}
\end{figure}
\begin{figure}[http]
	\centering
	\begin{tabular}{c}
		\includegraphics[width=.8\linewidth]{figures/recsinkc} \\ \includegraphics[width=.8\linewidth]{figures/recsinkb}
	\end{tabular}
	\caption{Thermal conduction with rectangular shape heat sink. $u_{min}=44.8$, $u_{max}=46.9$.}
	\label{fig:recsink}
\end{figure}
\begin{figure}[ht]
	\centering
	\begin{tabular}{c}
		\includegraphics[width=.8\linewidth]{figures/finsinkc} \\ \includegraphics[width=.8\linewidth]{figures/finsinkb}
	\end{tabular}
	\caption{Thermal conduction with fin shape heat sink. $u_{min}=34.9$, $u_{max}=40.1$.}
	\label{fig:finsink}
\end{figure}


\subsection{Numerical experiment of optimal control problem}
\subsubsection{Optimal control problem}
\quad In engineering, sometimes we want to know how much heat source provided to receive heat $u(x, t)$ in a physical domain $\Omega$ in a time period $[0, T]$ equals or approximates with $\hat{u}(x, t),\, (x, t)\in Q.$ We suppose that heat source following the form $F(x, t)=f(x, t)+q(x, t)$. This leads to optimize the functional
\begin{align}\label{J}
	J(q)=\frac{1}{2}\left\|u-\hat{u}\right\|_{L^2(Q)}^2+\frac{\gamma}{2}\left\|q\right\|_{L^2(Q)}^2,
\end{align}
where $q$ being the control variable and $\gamma>0$ being a regularization parameter.

To solve this problem, we use FreeFem++ software which provides an efficient tool called IPOPT. It is designed to perform optimal control problems, for more details see at \cite{}. To use this optimizer, we need to include the \textit{ff-Ipopt} dynamic library. The parameters including the objective function $J(f)$ and its gradient $\nabla J(f)$ following, see \cite{}
\begin{align}\label{gradJ}
	\nabla J(q)=z(x, t)+\gamma q(x, t),
	\end{align}
	where $z(x, t)$ is the solution of the adjoint problem
	\begin{align}\label{adjoint} 
	\begin{cases}
		-\frac{\partial z(x, t)}{\partial t}+\mathcal{L}z(x, t)=u-\hat{u}, & (x, t)\in Q,\\
		z(x, t)=0, & (x, t)\in S\\
		z(x, T)=0, & x\in \Omega.
	\end{cases}
\end{align}

\subsubsection{Numerical results}
\quad In this section, we experiment an example as in \cite{}. We use finite element method with $\theta=0.5$ known for Crank-Nicolson method to solve the direct problem \eqref{1.1} and adjoint problem \eqref{adjoint}. Consider $\Omega=(0, 1)^2$ and $T=0.1$ and homogeneous Dirichlet boundary condition. The right hand side $f$, the desired state $\hat{u}$ and the initial condition $u_0$ such that
\begin{align*}
	&f(x, t)=-\pi^4w_b(x, T),\\
	&\hat{u}(x, t)=\frac{b^2-5}{2+b}\pi^2 w_b(x, t)+2\pi^2w_b(x, T),\\
	&u_0(x)=\frac{-1}{2+b}\pi^2w_b(x, 0),
\end{align*}
where $w_b(x, t)=e^{b\pi^2 t}\sin(\pi x_1)\sin(\pi x_2), \; b\in \mathbb{R}$.

We chose the regularization parameter $\gamma=\pi^{-4}$ and the optimal solution triple $(\bar{q}, \bar{u}, \bar{z})$ of the optimal control problem \eqref{J} is given by
\begin{align*}
	&\bar{q}(x, t)=-\pi^4\left[w_b(x, t)-w_b(x, T)\right],\\
	&\bar{u}(x, t)=\frac{-1}{2+b}\pi^2w_b(x, t),\\
	&\bar{z}(x, t)=w_b(x, t)-w_b(x, T).
\end{align*}

First, we consider the behavior of the error for a sequence of discretization with decreasing size of the time steps and a fixed spatial triangulation with $N=1089$ nodes. Second, we examine the behavior of the error under refinement of the spatial triangulation for $M=1024$ time steps. We choose the free parameter $b$ to be $-\sqrt{5}$.

\begin{figure}[h!]
	\centering
	\includegraphics[width=\linewidth]{../freefem++/err_t}
	\caption{Refinement of the time steps for $N =1089$ spatial nodes}
\end{figure}

\begin{figure}[h!]
	\centering
	\includegraphics[width=\linewidth]{../freefem++/err_x}
	\caption{Refinement of the spatial triangulation for $M = 1024$ time steps}
\end{figure}



% --------------------------------------------------------------------
\section{Conclusion and perspective}
In this contribution, the heat equation is investigated with implicit time scheme and polynomial degree continuous Galerkin finite element method. From the numerical results, we establish that our method present a nice mesh convergence. So, it allows for reliable simulation during the heat transfer process and exhibits application of in thermal engineering. The solver can be extended to deal with other simulation of heat transfer or mechanical models. For this reason, the source code are available online at \url{https://github.com/hdnhan/seatuc2019/tree/master/freefem\%2B\%2B}. 
\\
It is noteworthy that the numerical results of present work is an important feature for our future research when we consider the heat problem in the context of optimal control problem \cite{HTLI14} or shape optimization, e.g \cite{TLP18}. 
% --------------------------------------------------------------------
\bibliographystyle{plain}
\bibliography{references}{}
\vfill

\end{document}
