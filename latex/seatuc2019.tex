% 6 pages max
\documentclass[twocolumn]{article}
% packages
%\usepackage[utf8]{vietnam}


% ---------------------------------------------
% perso : packages and macros
% ---------------------------------------------
\usepackage{latexsym}
\usepackage{mathrsfs}
%\usepackage{amsfonts}
\usepackage{amssymb}
\usepackage{hyperref}
\hypersetup{
	colorlinks=true, %colorise les liens
	breaklinks=true, %permet le retour a la ligne dans les liens trop longs
	urlcolor=blue, %couleur des hyperliens
	linkcolor=blue, %couleur des liens internes
	citecolor=blue, %couleur des references
	pdftitle={An implicit high order discontinuous
		Galerkin level set method
		for two-phase flow problems}, %informations apparaissant dans
	pdfauthor={Ta Pigeonneau Saramito}, %les informations du document
	pdfsubject={level set} %sous Acrobat.
}
\newcommand{\Frac}[2] {\displaystyle \frac{#1}{#2}}
\newcommand{\red}[1]  {{\color[rgb]{1,0,0}{{#1}}}}
\renewcommand{\leq}{\leqslant}
\renewcommand{\geq}{\geqslant}
\newcommand{\jump}[1]    {[\![{#1}]\!]}
\newcommand{\average}[1] {\{\!\!\{{#1}\}\!\!\}}




\usepackage{amsmath, amssymb, amsthm}
%\usepackage{color, graphicx, cases}
%\usepackage{hyperref}
%\hypersetup{
%	colorlinks=true,
%	linkcolor=black,
%	filecolor=black,      
%	urlcolor=black,
%}
%\usepackage{array, multirow, booktabs}
%\usepackage{caption, subcaption}
%\usepackage{ragged2e} % using justifying
\numberwithin{equation}{section}
\everymath{\displaystyle}
%\usepackage{titling}

%\usepackage[a4paper,left=35mm,top=31mm,right=20mm,bottom=30mm]{geometry}
%\renewcommand{\baselinestretch}{1.5}

% new definitions
\newtheorem{dl}{Theorem}[section]
\newtheorem{md}{Proposition}[section]
\newtheorem{hq}{Corollary}[section]
\newtheorem{cy}{Remark}[section]

\theoremstyle{definition}
\newtheorem{dn}{Definition}[section]
\newtheorem{vd}{Example}[section]
\newtheorem{bt}{Problem}[section]
\newtheorem{nx}{Comment}[section]

% reference
\usepackage{cleveref}
\crefname{dl}{\textbf{Theorem}}{}
\crefname{md}{\textbf{Proposition}}{}
\crefname{hq}{\textbf{Corollary}}{}

\crefname{dn}{\textbf{Definition}}{}
\crefname{vd}{\textbf{Example}}{}
\crefname{cy}{\textbf{Remark}}{}
\crefname{bt}{\textbf{Problems}}{}
\crefname{nx}{\textbf{Comment}}{}

\title{\bf Numerical simulation of heat transfer problem by Freefem++ software}
\author{Ta Thi Thanh Mai\thanks{email: abc}\and Ho Duc Nhan\thanks{email: abc}\and Tran Minh Tam\thanks{email: abc}}
%\thanksmarkseries{arabic}
\date{\footnotesize\textit{School of Applied Mathematics and Informatics, Hanoi University of Science and Technology, 1 Dai Co Viet street, Hai Ba Trung District, Hanoi, Vietnam}}
\justifying
% ---------------------------------------------
\begin{document}
\twocolumn[
	\maketitle
	\begin{@twocolumnfalse}
		\rule{\textwidth}{.1pt}
		\begin{abstract}
			content...
		\end{abstract}
		\vspace{1ex}
		\noindent \textit{Keywords: Inverse source problems, least squares method, Tikhonov regularization, conjugate gradient method.}
		\rule{\textwidth}{.1pt}
		\vspace{2mm}
	\end{@twocolumnfalse}
]
\saythanks

% --------------------------------------------------------------------
\section{Introduction}
% --------------------------------------------------------------------

Let $\Omega \subset \mathbb{R}^d,\, d\in \mathbb{N^+}$ be a bounded domain with a boundary $\Gamma$ and endow the cylinder $Q=\Omega\times (0,\, T]$ and lateral surface area $S=\Gamma \times (0,\, T]$ where $T>0$. 
\\
Consider the heat equation:
\begin{align}\label{1.1}
	\frac{\partial u(x, t)}{\partial t}+\mathcal{L}u(x, t)=F(x, t), \quad(x, t)\in Q,
\end{align}
with the Dirichlet boundary and initial conditions, respectively
\begin{align}
	u(x, t)&=u_D(x, t),\quad(x, t)\in S, \label{1.2}\\
	u(x, 0)&=u_0(x),\quad\quad\, x\in \Omega,\label{1.3}
\end{align}
where
\begin{align*}
	&\mathcal{L}u = -\sum_{i, j=1}^{d}\frac{\partial}{\partial x_i}\left(a_{ji}\frac{\partial u}{\partial x_j}\right)+a_0u,\\
	&a_{ji}\in L^{\infty}(Q),\, a_{ij}=a_{ji},\; \forall i, j\in \{1, 2, ..., d\},\\
	&\lambda_1\left\|\xi\right\|^2\leq \sum_{i, j=1}^{d}a_{ij}\xi_i\xi_j\leq \lambda_2\left\|\xi\right\|^2,\; \forall \xi\in\mathbb{R}^d,\\
	&a_0\in L^{\infty}(Q),\; 0\leq a_0(x, t)\leq \mu_1,\; (x, t)\in Q,\\ 
	&u_0\in L^2(\Omega),\;u_D\in L^2(S),
\end{align*}
with $\lambda_1$ and $\lambda_2$ are positive constants and $\mu_1\geq 0$.
\\
The problem is that to determine $u$ when all data $a_{ji},\,a_0,\,u_0,\,u_D$ and $F$ in \eqref{1.1} - \eqref{1.2} - \eqref{1.3} are given called \textbf{\textit{direct problem}}. 
%But in practice, we miss one of the data above such as  the right hand side $F$ of \eqref{1.1} known for heat source. The problem identifying $F$ when some additional observations on the solution $u$ available called \textbf{\textit{inverse problem}}. We suppose that the heat source following the form $F(x, t)=f(x, t)q(x, t)+g(x, t)$, where $q(x, t),\, g(x, t)$ are given. Find $f(x, t)$ if $\omega(x, t)=u(x, t)$ is given on $Q$. 
% --------------------------------------------------------------------
\section{Numerical method} 
% --------------------------------------------------------------------
\subsection{Variational problem}
% --------------------------------------------------------------------

Find $u(.,t)\in H^1(\Omega)$ such that
\begin{align}\label{2.1}
	\left\langle \frac{\partial u}{\partial t}, v \right\rangle+a\left(u, v\right)=\left\langle F, v \right\rangle,\; \forall v\in H^1(\Omega),
\end{align} 
\begin{align}\label{2.2}
	u(x, 0)=u_0(x), \; x\in \Omega,
\end{align}
where 
$$a\left(u, v\right)=\int_{\Omega}\left[\sum_{i, j=1}^{d}a_{ji}\frac{\partial u}{\partial x_i}\frac{\partial v}{\partial x_j}+a_0uv\right]dx,$$
$$\left\langle \varphi, v \right\rangle=\int_{\Omega}\varphi vdx.$$
\subsection{Finite element method}
Now we present a fully discrete finite element approximation for the variational problem \eqref{2.1} by the Crank-Nicolson method as follows:
\\
For \textit{spatial approximation}, let $\mathcal{T}_h$ be a triangulation of $\Omega$ and define a piecewise linear finite element space $V_h \subset H^1(\Omega)$ by
$$V_h=\left\{v_h:v_h\in C(\overline{\Omega}), v_h|_K\in P_1(K), \forall K\in \mathcal{T}_h\right\},$$
where $P_1(K)$ is a continuous piecewise linear polynomial on the element $K$. 
\\
For \textit{temporal discretization}, discrete $[0, T]$ uniformly into $M$ steps, $t_n=n\Delta t,\, n=0, 1, \dots, M$ with the time step size $\Delta t = T/M$. We define a function $\varphi(x, t)$ and $\varphi(x, t_n)=\varphi^n(x)$.
\\
Find $u^n_h\in V_h$ for $n=1, 2, \dots, M$ such that
\begin{align}\label{2.3}
	&\langle d_tu^n_h, v_h \rangle+a\left(\theta u^n_h+(1-\theta)u^{n-1}_h, v_h\right)\notag\\
	&\qquad\quad=\left\langle \theta F^n+(1-\theta)F^{n-1}, v_h \right\rangle, \; \forall v_h\in V_h,
\end{align}
and the initial condition 
\begin{align}\label{2.4}
	u^0_h=u_0,
\end{align}
where $d_tu^n_h=\frac{u^n_h-u^{n-1}_h}{\Delta t}, \; n=1, 2, ..., M.$

The discrete variational problem \eqref{2.3} admits a unique solution $u^n_h\in V_h$, see \eqref{}. Let $u_h(x, t)$ be the linear interpolation of $u_h^n$ with respect to $t$. 
For $x\in \Omega,\, t\in [t_{n-1}, t_n]$, we have
\begin{align*}
	u_h(x, t)=\frac{t-t_{n-1}}{\Delta t}u_h^{n-1}+\frac{t_n-t}{\Delta t}u_h^{n}.
\end{align*}
\begin{dl}\label{dl2.1}
	Let $u(x, t)$ be the solution of variational problem \eqref{2.1} - \eqref{2.2} and $u^n_h\in V_h$ for $n=1, 2, \dots, M$ be the solution for \eqref{2.3}. Then there holds the error estimate, see \eqref{}
	\begin{align}\label{2.5}
		\left\|u_h-u\right\|_{L^2(Q)}=O\left(h^2+\Delta t^2\right),
	\end{align}
	where $h$ is the mesh size.
\end{dl}



% --------------------------------------------------------------------
\section{Tests and discussion}
% --------------------------------------------------------------------
\subsection{Error evaluation with exact solution}
\quad We study a numerical experiment with the exact solution of heat equation and evaluate the error convergence. Consider a square $\Omega = [0,1]^2$. Find $u(x, t)$ satisfying
\begin{equation}
	\dfrac{\partial u}{\partial t} - \left(\dfrac{\partial^2 u}{\partial x_1^2} + \dfrac{\partial^2u}{\partial x_2^2}\right) = (1+2\pi^2)e^t\sin(\pi x_1) \sin(\pi x_2),
\end{equation}
with the initial and boundary conditions
$$ u(x, 0) = \sin(\pi x_1)\sin(\pi x_2) \quad \text{and} \quad u|_S = 0.$$
The exact solution is
$$u(x, t) = e^t\sin(\pi x_1) \sin(\pi x_2), $$

In order to quantify the difference between the approximate solution and the exact one, different cases of mesh size and time step length were studied to show the convergence of $L^2$-error. Fig. \ref{fig:time} shows, for this error measurement, the convergence of the approximate solution to the exact one when the time step tends to zero. Remark that, for each $h$, the error tends to a constant that is independent of time step. Numerically, we are looking for a representation of the error as the sum of two terms, one depends on $h$ and the other depend on $\Delta t$: $\left\|u_h-u\right\|_{L^2(Q)}=O\left(h^{\alpha}+\Delta t^{p}\right)$. Fig \ref{fig:mesh} shows the error versus $h$ for various mesh size, the time step $\Delta t$ has been chosen sufficiently small such that the error depends only upon $h$. Observe that the approximate solution converges to the exact one with mesh refinement with a power index $\alpha \approx 2$. This result is in good accordance with the Theorem \ref{dl2.1} in the previous section.
\begin{figure}[h!]
	\centering
	\begin{tabular}{c}
		\includegraphics[width=.8\linewidth]{figures/CNt}
	\end{tabular}
	\caption{$L^2$ error convergence of Crank-Nicolson scheme depends on time step.}
	\label{fig:time}
\end{figure}
\begin{figure}[h!]
	\centering
	\begin{tabular}{c}
		\includegraphics[width=.8\linewidth]{figures/CN}
	\end{tabular}
	\caption{$L^2$ error convergence of Crank-Nicolson scheme depends on mesh size.}
	\label{fig:mesh}
\end{figure}

\subsection{A problem of thermal engineering}
\quad The parameters $a_{ji}$ in \eqref{1.1} describe thermal conductivity property of a specific object. For a homogeneity object, thermal conductions are equal for all dimensions, which mean $a_{ji}=const \quad \forall i,j=\overline{1,d}$. The constant is thermal conductivity coefficient of the object material. In two dimensional case, \eqref{1.1} forms as
\begin{align}\label{4.2}
	\dfrac{\partial u}{\partial t} - \kappa \left(\dfrac{\partial^2 u}{\partial x_1^2}+\dfrac{\partial^2 u}{\partial x_2^2}\right) = F.
\end{align}
\quad We apply the numerical simulations of heat transfer into designing heat sink. Assuming a hot CPU inside a rectangular room fill with air. Let $u=u_{hot}$ inside CPU region and $u=u_{air}$ on air region, respectively $\Omega_{c}$ and $\Omega_{a}$, on the initial time. Our goal is to design a heat sink stick on the CPU to lower its temperature. Figure \ref{fig:testsink} describes the setting of this experiment.\\
\begin{figure}[ht]
	\centering
	\begin{tikzpicture}
		\draw (0,0) -- (6,0);
		\draw (6,0) -- (6,4);
		\draw (6,4) -- (0,4);
		\draw (0,4) -- (0,0);
		\draw (4,0) -- (4,0.8);
		\draw (4,0.8) -- (2,0.8);
		\draw (2,0.8) -- (2,0);
		\node at (3,0.4) {$\Omega_c$};
		\node at (3,2.5) {$\Omega_a$};
		\node at (3,1.1) {$\Omega_s$};
		\draw (2,1.5) .. controls (2.7,2) and (3.4,1) .. (4,1.2);
		\draw (2,0.8) -- (2,1.5);
		\draw (4,0.8) -- (4,1.2);
	\end{tikzpicture}
	\caption{Illustration of simple cooling system for CPU.}
	\label{fig:testsink}
\end{figure}
\newpage
The heat sink region, denoted by $\Omega_{s}$, has thermal conductivity coefficient $\kappa_s$. Similarly, let $\kappa_a$ and $\kappa_c$ be respectively the thermal conductivity coefficients inside air and CPU region. Technically, $\kappa_a$ value is small compared to $\kappa_c$ and $\kappa_s$ due to thermal conducting nature of air. Furthermore, to provide cooling ability, $\kappa_s>\kappa_c$. The visualization using medit software. At initial state, set $u_{hot}=80$ and $u_{air}=20$ as in Figure \ref{fig:ini}.\\
\begin{figure}[http]
	\centering
	\begin{tabular}{c}
		\includegraphics[width=.8\linewidth]{figures/begin}
	\end{tabular}
	\caption{Thermal distribution at initial state. $u|_{\Omega_{c}} = 80$, $u|_{\Omega_{a}\cup\Omega_{s}}=20$.}
	\label{fig:ini}
\end{figure}\\
Set $T=1s$, $\kappa_a=0.01$, $\kappa_c=1$, $\kappa_s=100$. The thermal distribution, maximum and minimum temperature inside domain $\Omega_{c}$  at final time $T$ of different heat sink shapes are shown in Figures \ref{fig:nosink}, \ref{fig:recsink} and \ref{fig:finsink}.\\
\begin{figure}[http]
	\centering
	\begin{tabular}{c}
		\includegraphics[width=.8\linewidth]{figures/nosinkc} \\ \includegraphics[width=.8\linewidth]{figures/nosinkb}
	\end{tabular}
	\caption{Thermal conduction without heat sink. $u_{min}=63.1$, $u_{max}=67.3$.}
	\label{fig:nosink}
\end{figure}
\begin{figure}[http]
	\centering
	\begin{tabular}{c}
		\includegraphics[width=.8\linewidth]{figures/recsinkc} \\ \includegraphics[width=.8\linewidth]{figures/recsinkb}
	\end{tabular}
	\caption{Thermal conduction with rectangular shape heat sink. $u_{min}=44.8$, $u_{max}=46.9$.}
	\label{fig:recsink}
\end{figure}
\begin{figure}[ht]
	\centering
	\begin{tabular}{c}
		\includegraphics[width=.8\linewidth]{figures/finsinkc} \\ \includegraphics[width=.8\linewidth]{figures/finsinkb}
	\end{tabular}
	\caption{Thermal conduction with fin shape heat sink. $u_{min}=34.9$, $u_{max}=40.1$.}
	\label{fig:finsink}
\end{figure}



% --------------------------------------------------------------------
\section{Conclusion}
% --------------------------------------------------------------------


% --------------------------------------------------------------------
% biblio
% --------------------------------------------------------------------
\bibliographystyle{plain}
\bibliography{references}{}
\vfill

\end{document}
