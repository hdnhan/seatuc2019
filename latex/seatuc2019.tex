% 6 pages max
\documentclass[notitlepage,a4paper,fleqn,9pt]{icmfarticle}
\input{icmfstyle.tex}
% packages
%\usepackage[utf8]{vietnam}


% ---------------------------------------------
% perso : packages and macros
% ---------------------------------------------
\usepackage{latexsym}
\usepackage{mathrsfs}
%\usepackage{amsfonts}
\usepackage{amssymb}
\usepackage{hyperref}
\hypersetup{
	colorlinks=true, %colorise les liens
	breaklinks=true, %permet le retour a la ligne dans les liens trop longs
	urlcolor=blue, %couleur des hyperliens
	linkcolor=blue, %couleur des liens internes
	citecolor=blue, %couleur des references
	pdftitle={An implicit high order discontinuous
		Galerkin level set method
		for two-phase flow problems}, %informations apparaissant dans
	pdfauthor={Ta Pigeonneau Saramito}, %les informations du document
	pdfsubject={level set} %sous Acrobat.
}
\newcommand{\Frac}[2] {\displaystyle \frac{#1}{#2}}
\newcommand{\red}[1]  {{\color[rgb]{1,0,0}{{#1}}}}
\renewcommand{\leq}{\leqslant}
\renewcommand{\geq}{\geqslant}
\newcommand{\jump}[1]    {[\![{#1}]\!]}
\newcommand{\average}[1] {\{\!\!\{{#1}\}\!\!\}}




\usepackage{amsmath, amssymb, amsthm}
%\usepackage{color, graphicx, cases}
%\usepackage{hyperref}
%\hypersetup{
%	colorlinks=true,
%	linkcolor=black,
%	filecolor=black,      
%	urlcolor=black,
%}
%\usepackage{array, multirow, booktabs}
%\usepackage{caption, subcaption}
%\usepackage{ragged2e} % using justifying
\numberwithin{equation}{section}
\everymath{\displaystyle}
%\usepackage{titling}

%\usepackage[a4paper,left=35mm,top=31mm,right=20mm,bottom=30mm]{geometry}
%\renewcommand{\baselinestretch}{1.5}

% new definitions
\newtheorem{dl}{Theorem}[section]
\newtheorem{md}{Proposition}[section]
\newtheorem{hq}{Corollary}[section]
\newtheorem{cy}{Remark}[section]

\theoremstyle{definition}
\newtheorem{dn}{Definition}[section]
\newtheorem{vd}{Example}[section]
\newtheorem{bt}{Problem}[section]
\newtheorem{nx}{Comment}[section]

% reference
\usepackage{cleveref}
\crefname{dl}{\textbf{Theorem}}{}
\crefname{md}{\textbf{Proposition}}{}
\crefname{hq}{\textbf{Corollary}}{}

\crefname{dn}{\textbf{Definition}}{}
\crefname{vd}{\textbf{Example}}{}
\crefname{cy}{\textbf{Remark}}{}
\crefname{bt}{\textbf{Problems}}{}
\crefname{nx}{\textbf{Comment}}{}
\everymath{\displaystyle}
\title{\bf 
  Numerical simulation of heat transfer problem by Freefem++ software
}
\author{\normalsize\bf
Mai Ta$^{1,2}$, Franck Pigeonneau and Pierre Saramito$^2$
}
\date{\normalsize\vspace{-2ex}\em
$^1$~Surface du verre et interfaces, UMR 125 CNRS/Saint-Gobain\\
39, quai Lucien Lefranc - BP 135, 93303 Aubervilliers cedex, France\\
\href{mailto:Franck.Pigeonneau@saint-gobain.com}{Franck.Pigeonneau@saint-gobain.com},\\[1.1mm]
%
$^2$~Lab. J. Kuntzmann, CNRS and Grenoble university\\
51, rue des math\'ematiques BP 53 - Domaine Universitaire, 38041 Grenoble Cedex 9, France \\
\href{mailto:tathithanhmai@gmail.com}{tathithanhmai@gmail.com},
\href{mailto:Pierre.Saramito@imag.fr}{Pierre.Saramito@imag.fr}
}

% ---------------------------------------------
\begin{document}

\raggedcolumns

\maketitle
\vspace{-3mm}
\noindent\rule{\textwidth}{.1pt}

\noindent
Abstract\\
\vspace{1ex}

\noindent
...
\\
\vspace{1ex}

\noindent {\em Keywords: Inverse source problems, least squares method, Tikhonov regularization, space-time finite element method, conjugate gradient method}

% ligne de separation
\noindent\rule{\textwidth}{.1pt}\vspace{2mm}

\begin{multicols}{2}
% --------------------------------------------------------------------
\section{Introduction}
% --------------------------------------------------------------------
%In mathematic and engineering, natural phenomena are described by PDEs. Following the current researches, a large amount of natural laws express through parabolic equation including the heat transfer problem. To be more detailed, 
Let $\Omega \subset \mathbb{R}^d,\, d\in \mathbb{N^+}$ be a bounded domain with a boundary $\Gamma$ and endow the cylinder $Q=\Omega\times (0,\, T]$ and lateral surface area $S=\Gamma \times (0,\, T]$ where $T>0$. 
\\
Consider the heat equation:
\begin{align}\label{1.1}
	\qquad\frac{\partial u(x, t)}{\partial t}+\mathcal{L}u(x, t)=F(x, t), \quad(x, t)\in Q,
\end{align}
with the Dirichlet boundary and initial conditions, respectively
\begin{align}
	\qquad\qquad u(x, t)&=u_D(x, t),\quad(x, t)\in S, \label{1.2}\\
	u(x, 0)&=u_0(x),\quad\quad\, x\in \Omega,\label{1.3}
\end{align}
where
\begin{align*}
	&\mathcal{L}u(x, t) = -\sum_{i, j=1}^{d}\frac{\partial}{\partial x_i}\left(a_{ji}(x, t)\frac{\partial u(x, t)}{\partial x_j}\right)+a_0(x, t)u(x, t)\\
	&a_{ji}\in L^{\infty}(Q),\, a_{ij}=a_{ji},\; \forall i, j\in \{1, 2, ..., d\},\\
	&\lambda_1\left\|\xi\right\|^2\leq \sum_{i, j=1}^{d}a_{ij}\xi_i\xi_j\leq \lambda_2\left\|\xi\right\|^2,\; \forall \xi\in\mathbb{R}^d,\\
	&a_0\in L^{\infty}(Q),\; 0\leq a_0(x, t)\leq \mu_1,\; (x, t)\in Q\\ 
	&u_0\in L^2(\Omega),\;u_D\in L^2(S),
\end{align*}
with $\lambda_1$ and $\lambda_2$ are positive constants and $\mu_1\geq 0$.
\\
The problem is that to determine $u$ when all data $a_{ji},\,a_0,\,u_0,\,u_D$ and $F$ in \eqref{1.1}-\eqref{1.2}-\eqref{1.3} are given called \textbf{\textit{direct problem}}. But in practice, we miss one of the data above such as  the right hand side $F$ of \eqref{1.1} known for heat source. The problem identifying $F$ when some additional observations on the solution $u$ available called \textbf{\textit{inverse problem}}. We suppose that the heat source following the form $F(x, t)=f(x, t)q(x, t)+g(x, t)$, where $q(x, t),\, g(x, t)$ are given. Find $f(x, t)$ if $\omega(x, t)=u(x, t)$ is given on $Q$. 
\\
Suppose that $F, f, g \in L^2(Q)$ and $ q\in L^\infty(Q)$ and hope to recover $f(x, t)$ from the observation $\omega(x, t)$. Since the solution $u(x, t)$ depends on the function $f(x, t)$, so we denote it by $u(x, t, f)$ or $u(f)$. Identify $f(x, t)$ satisfying 
$$u(f)=\omega(x, t).$$ We need to minimize the least square functional \cite{}
$$J_0(f)=\frac{1}{2}\left\|u(f)-\omega\right\|_{L^2(Q)}^2.$$
However, this minimization problem is unstable and there might be many minimizers to it. Therefore, we minimize the Tikhonov functional instead
\begin{align}\label{1.4}
	J_{\gamma}(f)=\frac{1}{2}\left\|u(f)-\omega\right\|_{L^2(Q)}^2+\frac{\gamma}{2}\left\|f-f^*\right\|_{L^2(Q)}^2,
\end{align}
where $\gamma>0$ being a regularization parameter, $f^*$ is an a prior estimation of $f$.
\\
Additionally, we use the space $W(0, T)$ define as
$$W(0, T)=\left\{u: u\in L^2(0, T; H^1(\Omega)), u_t\in L^2\left(0, T; H^{-1}(\Omega) \right)\right\}.$$
A weak solution in $W(0, T)$ of the equations \eqref{1.1}-\eqref{1.2} is a function $u(x, t)\in W(0, T)$ satisfying the identity
\begin{align}\label{1.5}
	\int_{Q}\left[\frac{\partial u}{\partial t}v+\sum_{i, j=1}^{d}a_{ji}\frac{\partial u}{\partial x_i}\frac{\partial v}{\partial x_j}+a_0uv\right]dxdt=\int_{Q}Fvdxdt,
\end{align}
and 
\begin{align}\label{1.6}
\qquad\qquad\qquad\qquad\; u(x, 0)=u_0,\; x\in \Omega,
\end{align}
where $ v \in L^2\left(0, T; H^1(\Omega)\right)$.
Following [...], we can prove that there exists a unique solution $u\in W(0, T)$ of the problem \cref{1.1,1.2,1.3}. 

% --------------------------------------------------------------------
%\section{Numerical method} 
% --------------------------------------------------------------------
\section{Direct problem}
We discrete $\Omega$ into finite elements with mesh $\mathcal{T}_h$ and define the piecewise linear finite element space $V_h \subset H^1(\Omega)$ by
$$V_h=\left\{v_h:v_h\in C(\overline{\Omega}), v_h|_K\in P_1(K), \forall K\in \mathcal{T}_k\right\},$$
where $P_1(K)$ is the space of linear polynomials on the element $K$. For fully discretization, we introduce a uniform partition of the integral $[0, T]:$
$$0=t_0<t_1<\cdots<t_M=T,$$ 
where $t_n=n\Delta t,\, n=0, 1, \dots, M$  with the time step size $\Delta t = T/M$.
\\
Let 
$$a^n(w, v)=\int_{\Omega}\sum_{i, j=1}^{d}a^n_{ji}(x)\frac{\partial w}{\partial x_i}\frac{\partial v}{\partial x_j}dx+\int_{\Omega}a_0^n(x)wvdx,$$
for $w, v\in H^1(\Omega)$. Define $f^n(x)=f(x, t_n)$ and $a^n(., .): H^1(\Omega)\times H^1(\Omega)\to \mathbb{R}$ is the bounded bilinear form and $H^1(\Omega)$-elliptic,
\begin{align}\label{elliptic}
	\qquad\qquad a^n(v, v)\geq c_a\left\|v\right\|_{H^1(\Omega)}^2, \; \forall v\in H^1(\Omega).
\end{align}
We now present the fully discrete finite element approximation for the variational problem \eqref{1.5} by the Crank-Nicolson method as follows: Find $u^n_h\in V_h$ for $n=1, 2, \dots, M$ such that
\begin{align}\label{2.2}
	\langle d_tu^n_h, v \rangle+a^n\left(\frac{u^n_h+u^{n-1}_h}{2}, v\right)=\left\langle \frac{F^n+F^{n-1}}{2}, v \right\rangle,
\end{align}
and
\begin{align}
	\qquad\qquad\qquad\langle u^0_h, v \rangle_{L^2(\Omega)}=\langle u^0, v \rangle_{L^2(\Omega)}
\end{align}
where $\langle.,.\rangle=\langle.,.\rangle_{L^2(\Omega)}$ and $d_tu^n_h=\frac{u^n_h-u^{n-1}_h}{\Delta t}, \; n=1, 2, ..., M.$
\\
The discrete variational problem \eqref{2.2} admits a unique solution $u^n_h\in V_h$. Let $u_h(x, t)$ be the linear interpolation of $u_h^n$ with respect to $t$. 
\begin{dl}\label{dl2.1}
	Let $u(x, t)$ be the solution of variational problem \eqref{1.5} - \eqref{1.6} and $u^n_h\in V_h$ for $n=1, 2, \dots, M$ be the solution for \eqref{2.2}. Then there holds the error estimate
	\begin{align}\label{2.4}
		\qquad\qquad\quad\left\|u_h-u\right\|_{L^2(Q)}=O\left(h^2+\Delta t^2\right),
	\end{align}
	where $h$ is the mesh size.
\end{dl}
\begin{proof}
	For $\phi\in H^1(\Omega)$, we define the elliptic projection $R_h:H^1(\Omega)\to V_h$ as the unique solution of the variational problem
	$$a(R_h\phi, v_h)=a(\phi, v_h), \; \forall v_h\in V_h.$$
	Denote $\left\|.\right\|_{H^s(\Omega)}=\left\|.\right\|_{s}$ and $\left\|.\right\|_{L^2(t_{n-1}, t_n;\,H^s(\Omega))}=\left\|.\right\|_{n,s}$.
	There holds the error estimate, see[...] with $s=\left\{1, 2\right\}$,
	\begin{align}\label{ritz}
		\qquad\qquad\;\left\|R_hv-v\right\|_0+h\left\|R_hv-v\right\|_1\leq ch^s\left\|v\right\|_s.
	\end{align}
	The idea here is split the error form into two terms following
	$$u_h-u=\underbrace{\left(u_h-R_hu\right)}_{\varphi}-\underbrace{\left(u-R_hu\right)}_{\rho}.$$
	In general, we consider $t=t_n$, therefore we have %$\varphi^n=u^n_h-R_hu^n$ and
	\begin{align*}
		\left\langle d_t\varphi^n, v\right\rangle &+\frac{1}{2}a(\varphi^n+\varphi^{n-1}, v)=\\
		&=\left\langle d_tu^n_h, v\right\rangle+\frac{1}{2}a(u^n_h+u^{n-1}_h, v)\\
		&\qquad\quad-\frac{1}{2}a(R_hu^n+R_hu^{n-1}, v)-\left\langle R_hd_tu^n, v\right\rangle\\		
		&=\frac{1}{2}\left \langle F^n+F^{n-1},v\right\rangle\\
		&\qquad\quad-\frac{1}{2}a(u^n+u^{n-1}, v)-\left\langle R_hd_tu^n, v\right\rangle\\
		&=\frac{1}{2}\left\langle \frac{\partial u^n}{\partial t}+\frac{\partial u^{n-1}}{\partial t}, v\right\rangle-\left\langle R_hd_tu^n, v\right\rangle\\
		%&=\frac{1}{2}\left\langle \frac{\partial u^n}{\partial t}+\frac{\partial u^{n-1}}{\partial t}-2d_tu^n, v\right\rangle+\left\langle d_tu^n-R_hd_tu^n, v\right\rangle\\
		&=\frac{1}{2}\left\langle \frac{\partial u^n}{\partial t}+\frac{\partial u^{n-1}}{\partial t}-2d_tu^n, v\right\rangle+\left\langle d_t\rho^n, v\right\rangle.
	\end{align*}
	Suppose that $v=\varphi^n+\varphi^{n-1}\in V_N$, we obtain
	\begin{align}
		&\left\langle d_t\varphi^n, \varphi^n+\varphi^{n-1}\right\rangle +\frac{1}{2}a(\varphi^n+\varphi^{n-1}, \varphi^n+\varphi^{n-1})\notag\\
		&\qquad\quad=\left\langle d_t\rho^n, \varphi^n+\varphi^{n-1}\right\rangle\notag\\
		&\qquad\qquad+\frac{1}{2}\left\langle \frac{\partial u^n}{\partial t}+\frac{\partial u^{n-1}}{\partial t}-2d_tu^n, \varphi^n+\varphi^{n-1}\right\rangle \label{ptss1}
	\end{align}
	In the right hand side of \eqref{ptss1}, we analyze two terms, the first is 
	$$\chi_{1,n}=d_t\rho^n=d_tu^n-R_hd_tu^n$$
	\begin{align*}
		\Rightarrow\left\|\chi_{1,n}\right\|_0&=\left\|d_tu^n-R_hd_tu^n\right\|_0\\
		&\leq ch^2\left\|d_tu^n\right\|_2
		=\frac{ch^2}{\Delta t}\left\|\int_{t_{n-1}}^{t_n}u_tds\right\|_2\\
		&\leq \frac{ch^2}{\Delta t}\sqrt{\int_{t_{n-1}}^{t_n}1^2dt}\;\sqrt{\int_{t_{n-1}}^{t_n} \left\|u_t\right\|_2^2ds}\\
		&=\frac{ch^2}{\sqrt{\Delta t}}\left\|u_t\right\|_{n, 2}.
	\end{align*}
	And the second is
	\begin{align*}
		\qquad\chi_{2,n}&=\frac{\partial u^n}{\partial t}+\frac{\partial u^{n-1}}{\partial t}-2d_tu^n\\
		&=u_t(t_n)+u_t(t_{n-1})-\frac{2}{\Delta t}\left[u(t_n)-u(t_{n-1})\right].
	\end{align*}
	We have Taylor expansion for a function $f(y),\; y=x + \delta x$
	$$f(y)=f(x)+\sum_{n=1}^{N}\frac{f^{(n)}(x)}{n!}(\delta x)^n+\int_{x}^{y}\frac{f^{(N+1)}(s)}{N!}(y-s)^Nds.$$
	Due to Taylor expansion, we have
	\begin{align*}
		&u_t(t_{n-1})=u_t^n-\Delta tu_{tt}^n+\int_{t_n}^{t_{n-1}} u_{ttt}(s)(t_{n-1}-s)ds,\\
		&u(t_{n-1})=u^n-\Delta tu_{t}^n+\frac{\Delta t^2}{2}u_{tt}^n+\int_{t_n}^{t_{n-1}} \frac{u_{ttt}(s)}{2}(t_{n-1}-s)^2ds.
	\end{align*}
	\begin{align*}
		\Rightarrow \chi_{2,n}&=2u^n_t-\Delta tu_{tt}^n+\int_{t_n}^{t_{n-1}} u_{ttt}(s)(t_{n-1}-s)ds\\
		&\quad-\left(2u_{t}^n-\Delta tu_{tt}^n-\int_{t_n}^{t_{n-1}} \frac{u_{ttt}(s)}{\Delta t}(t_{n-1}-s)^2ds\right)\\
		%&=\int_{t_{n-1}}^{t_n} (s-t_{n-1})u_{ttt}(s)ds-\frac{1}{\Delta t}\int_{t_{n-1}}^{t_n}(s-t_{n-1})^2u_{ttt}(s)ds\\
		&=\int_{t_{n-1}}^{t_n}\left(s-t_{n-1}-\frac{(s-t_{n-1})^2}{\Delta t}\right)u_{ttt}(s)ds.
	\end{align*}
	\begin{align*}
		&\Rightarrow\left\|\chi_{2, n}\right\|_0=\left\|\int_{t_{n-1}}^{t_n}\left(s-t_{n-1}-\frac{(s-t_{n-1})^2}{\Delta t}\right)u_{ttt}(.,s)ds\right\|_0\\
		&\qquad\leq\sqrt{\int_{t_{n-1}}^{t_n}\left(s-t_{n-1}-\frac{(s-t_{n-1})^2}{\Delta t}\right)^2ds}\;\sqrt{\int_{t_{n-1}}^{t_n}\left\|u_{ttt}\right\|^2_0ds }\\
		&\qquad=\sqrt{\frac{\Delta t^3}{30}}\;\left\|u_{ttt}\right\|_{n, 0}.
	\end{align*}
	Hence, we obtain
	\begin{align}\label{ptss2}
		&\left\langle \chi_{1, n}, \varphi^n+\varphi^{n-1}\right\rangle+\frac{1}{2}\left\langle \chi_{2, n}, \varphi^n+\varphi^{n-1}\right\rangle\notag\\
		&\quad\leq\left[\frac{ch^2}{\sqrt{\Delta t}}\left\|u_t\right\|_{n, 2}+\frac{1}{2}\sqrt{\frac{\Delta t^3}{30}}\;\left\|u_{ttt}\right\|_{n, 0}\right]\left\|\varphi^n+\varphi^{n-1}\right\|_0.
	\end{align} 
	On the other hand, due to \eqref{elliptic}, we estimate the left hand side of  (\ref{ptss1}) such that
	\begin{align}\label{ptss3}
		&\left\langle d_t\varphi^n, \varphi^n+\varphi^{n-1}\right\rangle +\frac{1}{2}a(\varphi^n+\varphi^{n-1}, \varphi^n+\varphi^{n-1})\notag\\
		&\qquad\geq\frac{1}{\Delta t}\left[\left\|\varphi^n\right\|^2_0-\left\|\varphi^{n-1}\right\|^2_0\right]+c_a\left\|\varphi^n+\varphi^{n-1}\right\|^2_0.
	\end{align}
	According to \eqref{ptss1} - \eqref{ptss2} - \eqref{ptss3} for $n=1, 2,\dots, M$, we have 
	\begin{align*}
		&\left\|\varphi^M\right\|^2_0-\left\|\varphi^{0}\right\|^2_0+2\Delta t\sum_{n=1}^{M}c_a\left\|\varphi^n+\varphi^{n-1}\right\|^2_0\\
		&\leq\sum_{n=1}^{M}\left[ch^2\left\|u_t\right\|_{n, 2}+\frac{\Delta t^2}{2\sqrt{30}}\;\left\|u_{ttt}\right\|_{n, 0}\right]\sqrt{\Delta t}\left\|\varphi^n+\varphi^{n-1}\right\|_0\\
		&\leq c\epsilon h^4\sum_{n=1}^{M}\left\|u_t\right\|_{n, 2}^2+\epsilon \Delta t\sum_{n=1}^{M}\left\|\varphi^n+\varphi^{n-1}\right\|^2_0\\
		&+\frac{\epsilon\Delta t^4}{120}\sum_{n=1}^{M}\left\|u_{ttt}\right\|_{n, 0}^2+\epsilon \Delta t\sum_{n=1}^{M}\left\|\varphi^n+\varphi^{n-1}\right\|^2_0.
	\end{align*}
	\begin{align*}
		&\Rightarrow\sum_{n=1}^{M}\Delta t\left\|\varphi^n+\varphi^{n-1}\right\|^2_0\\
		&\leq C h^4\left\|u_t\right\|_{L^2\left(0, T;\, H^2(\Omega)\right)}^2+C \Delta t^4\left\|u_{ttt}\right\|_{L^2(0, T;\, \Omega)}^2+\left\|\varphi^{0}\right\|^2_0\\
		&\leq C h^4\left\|u_t\right\|_{L^2\left(0, T;\, H^2(\Omega)\right)}^2+C \Delta t^4\left\|u_{ttt}\right\|_{L^2(0, T;\, \Omega)}^2+Ch^4\left\|u^{0}\right\|^2_2.
	\end{align*}
	For $x\in \Omega,\, t\in [t_{n-1}, t_n]$, we have
	\begin{align*}
		\qquad\quad&u_h(x, t)=\frac{t-t_{n-1}}{\Delta t}u_h^{n-1}+\frac{t_n-t}{\Delta t}u_h^{n},\\
		&R_hu(x, t)=\frac{t-t_{n-1}}{\Delta t}R_hu^{n-1}+\frac{t_n-t}{\Delta t}R_hu^{n}.
	\end{align*}
	\begin{align*}
		&\Rightarrow \int_{t_{n-1}}^{t_n}\left\|u_h-R_hu\right\|_0^2dt\\
		%&= \int_{t_{n-1}}^{t_n}\left\|\frac{t-t_{n-1}}{\Delta t}\left(u_h^{n-1}-R_hu^{n-1}\right)+\frac{t_n-t}{\Delta t}\left(u_h^{n}-R_hu^n\right)\right\|_0^2dt\\
		&= \int_{t_{n-1}}^{t_n}\left\|\frac{t-t_{n-1}}{\Delta t}\varphi^{n-1}+\frac{t_n-t}{\Delta t}\varphi^n\right\|_0^2dt\\%,\;\; \varphi^n=u_h^n-R_hu^n\\
		&=\frac{1}{4}\int_{t_{n-1}}^{t_n}\left\|\varphi^n+\varphi^{n-1}+\frac{t_n+t_{n-1}-2t}{\Delta t}\left(\varphi^n-\varphi^{n-1}\right)\right\|_0^2dt\\
		&\leq\frac{1}{4}\int_{t_{n-1}}^{t_n}\left\|\varphi^n+\varphi^{n-1}\right\|_0^2+\left\|\frac{t_n+t_{n-1}-2t}{\Delta t}\left(\varphi^n-\varphi^{n-1}\right)\right\|_0^2dt\\
		&\leq\frac{\Delta t}{4}\left\|\varphi^n+\varphi^{n-1}\right\|_0^2\\
		&\qquad+\frac{1}{4}\int_{t_{n-1}}^{t_n}\left(t_n+t_{n-1}-2t\right)^2dt\left\|d_tu_h^n-R_hd_tu_h^n\right\|^2_0\\
		&\leq\frac{\Delta t}{4}\left\|\varphi^n+\varphi^{n-1}\right\|_0^2+Ch^4\,\Delta t\left\|u_{h, t}\right\|_{n, 2}^2.
	\end{align*}
	Hence,
	\begin{align*}
		\int_0^T\left\|u_h-R_hu\right\|_0^2dt&\leq \frac{1}{4}\sum_{n=1}^{M}\Delta t\left\|\varphi^n+\varphi^{n-1}\right\|^2_0\\
		&+Ch^4\,\Delta t\left\|u_{h, t}\right\|_{L^2\left(0, T;\, H^2(\Omega)\right)}^2\\
		&=O(h^4+\Delta t^4).
	\end{align*}
	$$\Rightarrow \left\|u_h-R_hu\right\|_{L^2(Q)}=O(h^2+\Delta t^2).$$
	Due to \eqref{ritz}, we obtain
	$$\left\|R_hu-u\right\|_{L^2(Q)}=O(h^2+\Delta t^2).$$
	Therefore, the theory is proved.
\end{proof}

% --------------------------------------------------------------------
\section{Inverse problem}
\subsection{conjugate gradient method}
% --------------------------------------------------------------------
We will prove that $J_\gamma$ is Frechet differentiable and drive a formula for its gradient.
\begin{dl}
	The functional $J_\gamma$ is Frechet differentiable and its gradient $\nabla J_\gamma$ at $f$ has the form 
	\begin{align}\label{2.6}
		\qquad \nabla J_\gamma(f)=q(x, t)p(x, t)+\gamma \left(f(x, t)-f^*(x, t)\right),
	\end{align}
	where $p$(x, t) is the solution of the adjoint problem
	\begin{align}\label{2.5} 
		\qquad\quad\begin{cases}
		-\frac{\partial p(x, t)}{\partial t}+\mathcal{L}p(x, t)=u(f)-\omega, & (x, t)\in Q,\\
		u(x, t)=0, & (x, t)\in S\\
		p(x, T)=0, & x\in \Omega.
		\end{cases}
	\end{align}
\end{dl}
\noindent To find $f$ satisfied \eqref{1.4}, we use the conjugate gradient method (CG). Its iteration follows, we assume that at the $k$th iteration, we have $f^k$ and then the next iteration will be
$$f^{k+1}=f^k+\alpha_kd^k,$$
with
\begin{align*}
	\qquad\qquad\quad d^k&=\left\{\begin{array}{ll}
	-\nabla J_\gamma(f^k),& k=0,\\
	-\nabla J_\gamma(f^k)+\beta_kd^{k-1},& k>0,
	\end{array}\right.\\\\
	\beta_k&=\frac{\left\|\nabla J_\gamma (f^k)\right\|^2_{L^2(Q)}}{\left\|\nabla J_\gamma (f^{k-1})\right\|^2_{L^2(Q)}},
\end{align*}
and
$$\alpha_k=\operatorname*{arg\,min}_{\alpha\geq 0}J_\gamma(f^k+\alpha d^k).$$
To identify $\alpha_k$, we consider two problems
\begin{bt}\label{bt2.1}
	Denote the solution of this problem is $u[f]$
	\begin{align*}
		\qquad\begin{cases}
			\frac{\partial u(x, t)}{\partial t}+\mathcal{L}u(x, t)=f(x, t)q(x, t),&(x, t)\in Q,\\
			u(x, t)=0, & (x, t)\in S,\\
			u(x, 0)=0,&x\in \Omega.
		\end{cases}
	\end{align*}
\end{bt}
\begin{bt}\label{bt2.2}
	Denote the solution of this problem is $u(u_0, u_D)$
	\begin{align*}
		\qquad\begin{cases}
			\frac{\partial u(x, t)}{\partial t}+\mathcal{L}u(x, t)=g(x, t),&(x, t)\in Q,\\
			u(x, t)=u_D(x, t), & (x, t)\in S,\\
			u(x, 0)=u_0(x),&x\in \Omega.
		\end{cases}
	\end{align*}
\end{bt}
\noindent If do so, the observation operators have the form $u(f)= u[f]+ u(u_0, u_D)=Af+ u(u_0, u_D)$, with $A$ being bounded linear operator from $L^2(Q)$ to $L^2(Q)$.
We have
\begin{align*}
	&J_{\gamma}(f^k+\alpha d^k)=\frac{1}{2}\left\| u(f^k+\alpha d^k)-\omega\right\|_{L^2(Q)}^2+\frac{\gamma}{2}\left\|f^k+\alpha d^k-f^*\right\|_{L^2(Q)}^2\\[0.2cm]
	&=\frac{1}{2}\left\|\alpha Ad^k+Af^k+ u(u_0, u_D)-\omega\right\|_{L^2(Q)}^2+\frac{\gamma}{2}\left\|f^k+\alpha d^k-f^*\right\|_{L^2(Q)}^2\\[0.2cm]
	&=\frac{1}{2}\left\|\alpha Ad^k+ u(f^k)-\omega\right\|_{L^2(Q)}^2+\frac{\gamma}{2}\left\|f^k+\alpha d^k-f^*\right\|_{L^2(Q)}^2.
\end{align*}
Differentiating $J_\gamma(f^k+\alpha d^k)$ with respect to $\alpha$, we get
\begin{align*}
\frac{\partial J_\gamma(f^k+\alpha d^k)}{\partial \alpha} &= \alpha\left\|Ad^k \right\|_{L^2(Q)}^2+\left\langle Ad^k, u(f^k)-\omega\right\rangle_{L^2(Q)}\\[0.2cm]
&\quad+\gamma\alpha\left\| d^k\right\|_{L^2(Q)}^2+\gamma\left\langle d^k, f^k-f^*\right\rangle_{L^2(Q)}.
\end{align*}
Putting $\frac{\partial J_\gamma(f^k+\alpha d^k)}{\partial \alpha}=0$, we obtain
\begin{align*}
\alpha_k&=-\frac{\displaystyle\left\langle Ad^k,  u(f^k)-\omega\right\rangle_{L^2(Q)}+\gamma\left\langle d^k, f^k-f^*\right\rangle_{L^2(Q)}}{\displaystyle\left\|Ad^k\right\|^2_{L^2(Q)}+\gamma\left\|d^k\right\|^2_{L^2(Q)}}\\[0.2cm]
&=-\frac{\displaystyle\left\langle d^k, A^*\left( u(f^k)-\omega\right)\right\rangle_{L^2(Q)}+\gamma\left\langle d^k, f^k-f^*\right\rangle_{L^2(Q)}}{\displaystyle\left\|Ad^k\right\|^2_{L^2(Q)}+\gamma\left\|d^k\right\|^2_{L^2(Q)}}\\[0.2cm]
&=-\frac{\displaystyle\left\langle d^k, A^*\left( u(f^k)-\omega\right)+\gamma(f^k-f^*)\right\rangle_{L^2(Q)}}{\displaystyle\left\|Ad^k\right\|^2_{L^2(Q)}+\gamma\left\|d^k\right\|^2_{L^2(Q)}}\\[0.2cm]
&=-\frac{\left\langle d^k,\nabla J_\gamma(f^k)\right\rangle_{L^2(Q)}}{\displaystyle\left\|Ad^k\right\|^2_{L^2(Q)}+\gamma\left\|d^k\right\|^2_{L^2(Q)}}.
\end{align*}
Because of $d^k=r^k+\beta_kd^{k-1},\, r^k=-\nabla J_\gamma (f^k)$ and $\left\langle r^k,d^{k-1}\right\rangle_{L^2(Q)}=0$, we get 
$$\alpha_k=\frac{\left\|r^k\right\|^2_{L^2(Q)}}{\displaystyle\left\|Ad^k\right\|^2_{L^2(Q)}+\gamma\left\|d^k\right\|^2_{L^2(Q)}}.$$
Thus, the CG algorithm is set up by following loop:

\noindent \textbf{CG algorithm}
\begin{itemize}
	\item[1.] Set $k=0$, initiate $f^0$.
	\item[2.] For $k=0, 1, 2,...$. Calculate
	$$r^k=-\nabla J_\gamma(f^k).$$
	Update\\
	\begin{align*}
	d^k&=\left\{\begin{array}{ll}
	r^k,& k=0,\\
	r^k+\beta_kd^{k-1},& k>0,
	\end{array}\right.\\\\
	\beta_k&=\frac{\left\|r^k\right\|^2_{L^2(Q)}}{\left\|r^{k-1}\right\|^2_{L^2(Q)}}.
	\end{align*}
	\item[3.] Calculate
	$$\alpha_k=\frac{\left\|r^k\right\|^2_{L^2(Q)}}{\displaystyle\left\|Ad^k\right\|^2_{L^2(Q)}+\gamma\left\|d^k\right\|^2_{L^2(Q)}}.$$
	Update
	$$f^{k+1}=f^{k}+\alpha_kd^k.$$
\end{itemize}

\subsection{Finite element discretization}
We rewrite the Tikhonov functional
\begin{align*}
J_\gamma(f)&=\frac{1}{2}\left\| u[f]+ u(u_0, u_D)-\omega\right\|^2_{L^2(Q)}+\frac{\gamma}{2}\left\|f-f^*\right\|^2_{L^2(Q)}\\
&=\frac{1}{2}\left\|Af-\hat{\omega}\right\|^2_{L^2(Q)}+\frac{\gamma}{2}\left\|f-f^*\right\|^2_{L^2(Q)},
\end{align*}
with $\hat{\omega}=\omega- u(u_0, u_D)$.
\\
The solution $f^\gamma$ of the minimization problem \eqref{2.4} is characterized by the first-order optimality condition
\begin{align}\label{3.1}
\nabla J_\gamma(f^\gamma)= A^*(Af^\gamma-\hat{\omega})+\gamma(f^\gamma-f^*)=0,
\end{align}
with $A^*: L^2(Q)\to L^2(Q)$ is the adjoint operator of $A$ defined by $A^*\left( u(f) - \omega\right) = p$ where $p$ is the solution of the adjoint problem \eqref{2.5}. 
\\
We will approximate \eqref{3.1} by finite element method. In fact, we will approximate $A$ and $A^*$ as follows.

\subsection{Approximation of $A,\, A^*$}
We use the fully discrete finite element approximation for the variational problem \eqref{2.1} by the Crank-Nicolson method. Hence, the discrete version of the optimal control problem \eqref{2.4} will be
$$J_{\gamma, h}(f)=\frac{1}{2}\left\|A_hf-\hat{\omega}_h\right\|^2_{L^2(Q)}+\frac{\gamma}{2}\left\|f-f^*\right\|^2_{L^2(Q)}\to \min.$$
Let $f^\gamma_h$ be the solution of this problem is characterized by the variational equation
\begin{align}\label{3.3}
\nabla J_{\gamma, h}(f^\gamma_h)= A_h^*(A_hf^\gamma-\hat{\omega}_h)+\gamma(f^\gamma_h-f^*)=0,
\end{align}
where $A_h^*$ is the adjoint operator of $A_h$. But it is hardly to find $A^*_h$ from $A_h$ in practice, so we define a proximate $\hat{A}_h^*$ of $A^*$ instead. Therefore, we suppose that $\hat{A}^*_h\phi=p_h$, where $\phi= u(f) - \omega$ and $p_h$ is the approximate solution of adjoint problem \eqref{2.5}. Therefore, the equation above will be
\begin{align}\label{3.4}
\nabla J_{\gamma, h}(f^\gamma_h)\simeq\nabla J_{\gamma, h}(\hat{f}^\gamma_h)= \hat{A}_h^*(A_h\hat{f}^\gamma-\hat{\omega}_h)+\gamma(\hat{f}^\gamma_h-f^*)=0,
\end{align}
Moreover, the observation will have noise in practice, so instead of $\omega$, we suppose that only get $\omega^{\delta}$ satisfying
$$\left\| \omega-\omega^\delta\right\|_{L^2(Q)}\leq \delta.$$
Therefore, instead of getting $\hat{f}^\gamma_h$ that satisfies the equation \eqref{3.5}, we will get $\hat{f}^{\gamma, \delta}_h$ satisfying
\begin{align}\label{3.5}
\nabla J_{\gamma, h}\left(\hat{f}^{\gamma, \delta}_h\right)= \hat{A}_h^*(A_h\hat{f}^{\gamma, \delta}_h-\hat{\omega}_h^\delta)+\gamma(\hat{f}^{\gamma, \delta}_h-f^*)=0,
\end{align}
with $\hat{\omega}_h^\delta=\omega^\delta- u_h(u_0, u_D)$.

\begin{dl}\label{dl3.3}
	Let $f^\gamma$ and $\hat{f}^\gamma_h$ are the solution of variational problems \eqref{3.1} and \eqref{3.4}, respectively. Then there hold a error estimate
	\begin{align}\label{3.11}
	\left\|f^\gamma-\hat{f}^\gamma_h \right\|_{L^2(Q)}\leq c(h^2+\Delta t^2).
	\end{align}
\end{dl}

\begin{cy}\label{cy3.1}
	Let $f^\gamma$ and $\hat{f}^\gamma_h$ are the solution of variational problems \eqref{3.1} and \eqref{3.5}, respectively. Then there hold a error estimate
	\begin{align}\label{3.12}
	\left\|f^\gamma-\hat{f}^{\gamma, \delta}_h \right\|_{L^2(Q)}\leq c(h^2+\Delta t^2+\delta).
	\end{align}
\end{cy}


% --------------------------------------------------------------------
\section{Tests and discussion}
% --------------------------------------------------------------------
\subsection{Exact solution}
% --------------------------------------------------------------------



% --------------------------------------------------------------------
\subsection{A problem of thermal engineering}
% --------------------------------------------------------------------



% --------------------------------------------------------------------
\subsection{Numerical experiment of inverse problem}
% --------------------------------------------------------------------



% --------------------------------------------------------------------
\section{Conclusion}
% --------------------------------------------------------------------


% --------------------------------------------------------------------
% biblio
% --------------------------------------------------------------------
\bibliographystyle{plain}
\bibliography{references}{}
\vfill
\end{multicols}
\end{document}
